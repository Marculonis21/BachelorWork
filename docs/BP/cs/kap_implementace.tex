%%% Fiktivní kapitola s ukázkami tabulek, obrázků a kódu

\chapter{Implementace projektu}
V předchozí kapitole jsme prošli funkční požadavky, očekávané od vyvíjeného
souboru programů. Následuje rozbor jednotlivých modulů, které vznikly při
vlastní implementaci. 

Celý projekt je napsána v programovacím jazyce \textbf{Python}. Cílem projektu
je vytvořit čitelnou rozšířitelnou platformu, která bude uživateli jednoduše
dostupná. Pokud uživatel bude mít potřebu vytvořené moduly jakkoli měnit nebo
rozšiřovat, Python toto bez problémů umožní. Jednoduchá čitelnost Pythonu
spojená s rychlostí, jakou mohou být prováděny iterace změn, bez potřeby
zdlouhavého překladu celé knihovny, se nám zdají býti dostatečně užitečné
vlastnosti volbu Pythonu jako jazyka pro tento projekt.

\paragraph{}
Dále v této kapitole v sekci \ref{imp:roboevo} popíšeme centrální modul
\emph{RoboEvo}, pomocí kterého může uživatel s knihovnou pracovat z příkazové
řádky nebo ze samotného kódu. Tento modul pracuje s řadou dalších
pomocných modulů, jejichž implementace popíšeme v dalších oddílech. Třídy
agentů popisující celé evoluční algoritmy v oddílu \ref{imp:gaAgents}.
Implementace vlastních genetických operátorů v oddílu \ref{imp:gaMethods}.
Třída propojující roboty ze simulátoru \emph{MuJoCo} (MuJoCo popsáno v
základních pojmech v oddílu \ref{MuJoCo}) s ostatními třídami (v oddílu
\ref{imp:robots}). Následovně si v sekci \ref{imp:GUI} popíšeme architekturu
grafické aplikace, umožňující nastavování experimentů v centrálním modulu bez
použití kódu nebo příkazové řádky. Jako poslední (v sekci
\ref{imp:experimentsetter}) si představíme implementaci třídy sloužící k
uchování a předvolbě parametrů pro experimenty.

\section{Modul RoboEvo} \label{imp:roboevo}
Modul \emph{RoboEvo} je centrální modul tohoto projektu, sloužící hlavně pro
spouštění a běh experimentů s evolučním vývojem robotů. 

Každý experiment se skládá vždy z několika nezávislých bloků. Experiment může
využívat různé typy evolučních agentů s různými genetickými operátor a může se
snažit vyvíjet různé typy robotů. Pro usnadnění přehlednosti, čitelnosti a
rozšířitelnosti jsou tyto části rozdělené do vlastních menších implementací,
rozšiřující hlavní modul (jednotlivé implementace budou popsány v dalších
oddílech). 

\paragraph{Implementace modulu \emph{RoboEvo}}
Implementace modulu se snazší co nejvíce oddělit funkční jádro modulu (funkce,
využívané převážně v průběhu evolučního algoritmu a uživatel do nich nemusí
zasahovat) od interaktivní části programu, pomocí které může uživatel (buď v
kódu nebo pomocí argumentů z příkazové řádky) konfigurovat a spouštět
experimenty. 

Hlavní funkcí ve které probíhá samotná konfigurace parametrů pro experimenty je
funkce \texttt{main}, která je spuštěná první po vstupu uživatele do aplikace.
Zde modul vyhodnocuje, zda uživatel při spuštění nevybral nějaký z nepovinných
parametrů, které by ovlivnily konfiguraci a spuštění experimentu. 

\pagebreak
Těmito parametry jsou:
\begin{itemize}
    \item \texttt{-{}-experiment} -- argument přijímající textový vstup
        specifikující jméno experimentu, jehož parametry chceme načíst a
        spustit (pracující s modulem \texttt{experiment\_setter} popsaného v
        oddíle \ref{imp:experimentsetter}),
    \item \texttt{-{}-experiment\_names} -- při výběru tohoto argumentu při
        spuštění programu program vypíše názvy všech dostupných vytvořených
        experimentů z modulu \texttt{experiment\_setter} a následně se ukončí,
    \item \texttt{-{}-batch} -- argument přijímající číselnou hodnotu,
        specifikující kolikrát se má nakonfigurovaný experiment opakovaně
        spustit (používané pro statistické vyhodnocení výsledků experimentů),
    \item \texttt{-{}-batch\_note} -- textový argument umožňující připojit
        vlastní poznámku k názvu složky, do které se experimenty z
        několikanásobného spuštění ukládají (argument nemá žádný efekt pro
        experimenty z modulu \texttt{experiment\_setter}),
    \item \texttt{-{}-open} -- textový argument přijímající cestu k uloženým
        datům nejlepšího jedince z libovolného předchozího experimentu,
        umožňující vizualizaci řešení daného jedince.
\end{itemize}

Pokud uživatel zvolil spuštění buď jednorázového, nebo opakovaného experimentu,
funkce \texttt{main} pro spouštění evolučního vždy vývoje volá funkci
\texttt{RunEvolution}, parametry experimentu jako je zvolený typ genetického
agenta, typ robota, parametry evolučního algoritmu a další.

Tato funkce zajišťuje vše od spuštění výpočetních jednotek a samotného
evolučního algoritmu se zvolenými parametry, po prezentaci a uložení
výsledných dat (při zvolení opakovaného experimentu je tato funkce volána
opakovaně).

\paragraph{Běh evolučního algoritmu}
Funkce \texttt{RunEvolution} zajišťuje spuštění evolučního algoritmu s danými
parametry. Samotný běh evolučního algoritmu je spuštěn voláním funkce
\texttt{evolution}, ve které se celý algoritmus bude odehrávat. V rámci této
funkce provádíme všechny kroky evolučního algoritmu (jak byly popsány v
základních pojmech evolučních algoritmů v sekci \ref{Evoluční algoritmy}).
Navíc zde pro jedince vytváříme simulační prostředí, ve kterých jsou jedinci
testování při výpočtu fitness.

Po výpočtu fitness přichází na řadu genetické operátory, které jsou vždy
specifické pro zvolený evoluční algoritmus. V naší implementaci jsou zvolené
operátory specifikované v třídě agenta. Ta mimo jiné specifikuje jak vypadá
genotyp jedinců, jakým způsobem se generuje populace jedinců a jak probíhá
rozkódování genotypu každého jedince do aktuální konfigurace motorů ovládaného
robota. Podrobněji popíšeme třídu agentů v dalším oddíle \ref{imp:gaAgents}.

\subsection{Agenti} \label{imp:gaAgents}

\subsection{Genetické operátory} \label{imp:gaMethods}

\subsection{Roboti} \label{imp:robots}

\section{Grafické rozhraní} \label{imp:GUI}

\section{Třída experimentů} \label{imp:experimentsetter}

