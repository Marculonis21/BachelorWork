%%% Fiktivní kapitola s ukázkami tabulek, obrázků a kódu

\chapter{Implementace}
V předchozí kapitole jsme prošli funkční požadavky, očekávané od vyvíjeného
souboru programů. Následuje rozbor jednotlivých modulů, které vznikly při
vlastní implementaci. Zároveň zde projdeme možné alternativy, které se pro
vývoj nabízejí a probereme důvody stojící za zvolením jednotlivých z možností.

Nejprve vysvětlíme volbu programovacího jazyka, ve kterém je celá knihovna
vytvořena. Poté projdeme systémy umožňující vývoj ve fyzikálním prostředí a
ovládání uživatelem definovaných robotů. Zde představíme i možnosti tvorby
vlastních robotů. Dále ukážeme možné varianty modulů umožňující vývoj řízení
robotů pomocí genetických algoritmů a popíšeme vlastní implementaci. Následně
projdeme všechny části implementace spojující tyto moduly do přístupné
rozšířitelné knihovny. V poslední části představíme implementaci grafického
rozhraní, které slouží uživateli, který chce používat knihovnu a provádět
experimenty, bez nutnosti využití příkazové řádky.

\section{Programovací jazyk}


\section{Simulované fyzikální prostředí}

Jelikož chceme vyvíjet řízení robotů založených na korektních fyzikálních
pravidlech a interakcích, je pro tuto práci důležité vybrat dostatečně
robustní, deterministický fyzikální simulátor. Dále bychom od tohoto simulátoru
chtěli, abychom~mohli dle vlastních potřeb měnit vlastnosti a podobu simulovaného
prostředí. Zároveň chceme, aby nám simulátor dovolil konfigurovat morfologii
vlastních robotů a případně nějakým stylem umožnil morfologii v průběhu běhu
vývoje měnit. V~poslední řadě by bylo užitečné, aby modul spravující zvolený
fyzikální simulátor byl open-source, což nám dá volnost v případě, že si budeme
chtít chování systémů v prostředí nějak vlastnoručně upravit.

Při hledání fyzikálních simulátorů, které by umožňovali kontrolu a ovládání
prostředí skrz Python jsme narazili na několik možností.

\begin{itemize}
    \item \textbf{Gazebo}\\
        Gazebo je sada open-source knihoven pro vývoj, výzkum a aplikaci
        robotů. Umožňuje simulaci dynamického 3D prostředí s více agenty,
        generování dat ze simulovaných senzorů a fyzikálně korektní interakce
        robotů s prostředím. Uživatel s knihovnou pracuje skrz grafické
        rozhraní nebo příkazovou řádku. Prostředí a roboti mohou být tvořené
        buď  skrz grafické prostředí, nebo v textovém formátu XML.
        \citet{koenig2004design}. 
    \item \textbf{Webots}\\
        Webots je open-source víceplatformní robotický simulátor, umožňující
        programování a testování virtuálních robotů a následnou aplikaci
        softwaru na reálné roboty. Využívá programovacích jazyků C a C++ s
        možností přístupu skrz Python. Prostředí dále nabízí využití
        připravených modelů robotů a možnosti vložení vlastních robotů z 3D
        modelovacích softwarů v CAD formátu \citet{michel2004cyberbotics}. Více
        informací v dokumentaci Webots \citep{Webots}.

\end{itemize}



\subsection{Roboti}

\section{Genetické algoritmy}

\section{Implementace knihovny}

\section{Grafické rozhraní}
