%%% Fiktivní kapitola s ukázkami tabulek, obrázků a kódu

\chapter{Implementace}
V předchozí kapitole jsme prošli funkční požadavky, očekávané od vyvíjeného
souboru programů. Následuje rozbor jednotlivých modulů, které vznikly při
vlastní implementaci. Zároveň zde projdeme možné alternativy, které se pro
vývoj nabízejí a probereme důvody stojící za zvolením jednotlivých z možností.

Nejprve vysvětlíme volbu programovacího jazyka, ve kterém je celá knihovna
vytvořena. Poté projdeme systémy umožňující vývoj ve fyzikálním prostředí a
ovládání uživatelem definovaných robotů. Zde představíme i možnosti tvorby
vlastních robotů. Dále ukážeme možné varianty modulů umožňující vývoj řízení
robotů pomocí genetických algoritmů a popíšeme vlastní implementaci. Následně
projdeme všechny části implementace spojující tyto moduly do přístupné
rozšířitelné knihovny. V poslední části představíme implementaci grafického
rozhraní, které slouží uživateli, který chce používat knihovnu a provádět
experimenty, bez nutnosti využití příkazové řádky.

\section{Programovací jazyk}

Jako programovací jazyk, ve kterém tento projekt bude psán, jsme zvolili\\
\textbf{Python}. Cílem projektu je vytvořit platformu, kterou bude uživatel
moci použít k vývoji robotů pomocí evolučních algoritmů. Pokud uživatel bude
mít potřebu jakkoli připravený proces vývoje měnit, Python lehce umožní
nahlédnout do zdrojových kódů vypracované knihovny a provést úpravy dle
vlastních potřeb. Zároveň to umožňuje rozšiřování knihovny o nové metody, které
bude chtít uživatel zkusit zařadit do již funkčního procesu. Jednoduchá
čitelnost Pythonu spojená s rychlostí, jakou mohou být prováděny iterace změn
bez potřeby zdlouhavého překladu celé knihovny, se~zdají býti dostatečně dobré
vlastnosti pro volbu programovacího jazyka pro tento projekt.

\section{Simulované fyzikální prostředí}

Po zhodnocení vypsaných a dalších možností jsme vybrali pro využití v tomto
projektu fyzikální simulátor MuJoCo. Na rozdíl od ostatních se zdá býti
přístupnější do začátku a zároveň dostatečně robustní a konfigurovatelný tak,
aby splnil veškeré požadavky, které od fyzikálního simulátoru máme.

\textbf{MuJoCo 1.50} je fyzikální simulátor zpřístupněný skrz volně dostupné
aplikační rozhraní společnosti \textbf{OpenAI} v rámci jejich sady různých
prostředí \textbf{Gym} (textové hry, jednoduché 2D i plně fyzikálně simulované
3D prostředí, Atari hry aj.) pro vývoj metod zpětnovazebného učení na různých
problémech. Toto rozhraní umožňuje uživatelům jednoduchý přístup k datům z
poskytnutých prostředí a ovládání prostředím definovaných agentů, pomocí
standardizovaných vstupů i výstupů napříč všemi prostředími. Tímto způsobem
můžeme velmi lehce ovládat i roboty v prostředích simulátoru MuJoCo. Navíc
otevřená vlastnost tohoto aplikačního rozhraní umožňuje úpravu částí procesu
tak, aby se lépe hodil při řešení námi zvolených problémů. Přestože je
\textbf{Gym} převážně používána pro vývoj metod zpětnovazebného učení agentů,
nic nám nebrání a je velmi jednoduché namísto toho využít vlastního agenta,
který je vyvíjen pomocí evolučních algoritmů.

\subsection{Roboti}

\section{Genetické algoritmy}

Po porovnání různých možností modulů pro tvorbu a použití evolučních algoritmů
v naší knihovně, jsme se rozhodli pro vlastní implementaci jednotlivých částí
evolučních algoritmů (genetických operátorů) a jejich propojení mezi sebou.
Důvodem je hlavně jednodušší zapojení do zbytku knihovny a snížení nároků na
znalosti mnohdy složitých výše popsaných externích knihoven pro uživatele,
který by případně mohl chtít si do naší knihovny dopsat vlastní kus evolučního
algoritmu. Tímto způsobem, pokud bude chtít něco takového udělat, dojde-li
k~dodržení zdrojovým kódem stanovených pravidel, vlastní kus kódu (metoda
popisující genetický operátor) bude možné hned bez problému využít při dalším
vývoji robotů.

\section{Implementace knihovny}

\section{Grafické rozhraní}
