\chapter{Specifikace} \label{chapter-specifikace}

Vývoj pomocí evolučních algoritmů je možné nejlépe ilustrovat pomocí
experimentů, na kterých může uživatel sám pozorovat změny, kterými postupný
iterativní evoluční vývoj nachází možná řešení pro zadaný problém. Je ale
složité vytvořit takový systém, ve kterém by uživatel mohl snadno ovládat
interní části evolučních algoritmů, a tak vytvářet vlastní různorodé
experimenty. A právě tyto experimenty mohou být zásadní pro pochopení
specifických zákoutí aplikace evolučních algoritmů.

Proto cílem tohoto projektu je návrh knihovny, která by uživatelům,
přicházejících z různých oborů, umožnila bližší pochopení a seznámení se s
evolučními algoritmy pomocí vlastních interaktivních experimentů při vývoji
robotů v simulovaném prostředí. 

I jednoduché problémy, které od robotů můžeme požadovat vyřešit (např. ujití co
největší možné vzdálenosti za daný čas), poskytují pro roboty různé složitosti
(různé morfologie, počtu kloubů atd.) dobrou představu o nárůstu obtížnosti
daného problému. Tímto zároveň experimenty s různými roboty vynucují využití
různých pokročilejších metod pro dosažení požadovaných cílů daného experimentu.

V následující sekci \ref{Specifikace-funkčnípožadavky} zabývající se funkčními
požadavky si představíme jednotlivé vlastnosti, které od takového systému
budeme požadovat.

\section{Funkční požadavky} \label{Specifikace-funkčnípožadavky}

Cílem tohoto projektu je vytvořit systém umožňující uživatelům vytvářet vlastní
experimenty s evolučním vývojem robotů v simulovaném fyzikálním prostředí.
Uživatel by měl být schopný před spuštěním experimentu podrobně pochopit a
upravit co nejvíce částí evolučního vývoje, který bude v době experimentu
probíhat. Uživatel musí být v době běhu experimentu schopný sledovat průběžné
výsledky z~jednotlivých generací a vizualizovat dosavadní výsledky v
simulovaném prostředí. Po dokončení experimentu musí být možné uložit výsledky
ve formě dále zpracovatelné, např. pro statistický rozbor většího množství
experimentů s~možností vizualizace dat nejlepších jedinců finálních generací a
podobně.

Systém bude z hlavní části vytvořený v programovacím jazyce Python, vytvářející
uživateli přístupnější kód a umožňující rychlejší experimentování a
prototypování nápadů. Python je vhodný, jelikož se pro tento systém nesnažíme
o~maximální efektivitu nebo rychlost experimentů, ale o čitelnost celého
systému a schopnost vytvářet s naší knihovnou vlastní experimenty. Zároveň nám
to umožní zpřístupnit otestovaný program uživatelům na operačních systémech
Windows i Linux.

Univerzálnost navrženého řešení umožní uživateli přistupovat k našemu systému
třemi způsoby popsanými níže. Každý způsob přístupu k systému má své vlastní
požadavky, které s sebou přináší. 

\paragraph{Grafické rozhraní}
Pro uživatele, kteří preferují jednoduchý přístup zprostředkovaný interaktivním
grafickým rozhraním, musí systém umožňovat konfigurovat experimenty dostatečné
složitosti z prostředí tohoto grafického rozhraní. Uživatel tímto způsobem bude
dále schopný pozorovat průběžné výsledky evolučního vývoje a vizualizovat
průběžná nejlepší řešení, které evoluční vývoj najde.

\paragraph{Python knihovna}
Pro uživatele, kteří chtějí navrhovat vlastní experimenty, ale nechtějí všechno
programovat od základů bude, projekt dostupný ve formě knihovny. Systém pro
tvorbu experimentů vývoje robotů v simulovaném prostředí bude tvořit otevřenou
Python knihovnu, kterou uživatel může připojit ke svému projektu a pomocí naší
knihovny jednoduše vytvářet experimenty s požadovanou volností konfigurace
jednotlivých parametrů evolučního vývoje.

Knihovna bude mít dostatečnou dokumentaci na to, aby takový uživatel byl
schopen provádět pokročilou konfiguraci experimentů přímo v kódu knihovny,
a~aby tyto experimenty bylo možné provádět z kódu, bez omezení výstupů systému
nebo vizualizace řešení.

\paragraph{Rozšiřování knihovny}
Pro pokročilé uživatele bude navržený systém rozšiřitelnou platformou. Dokumentace
představí a vysvětlí technologie využité při vývoji této knihovny a kód
knihovny bude sestavený tak, aby byl dobře přístupný a jednoduše rozšířitelný.
