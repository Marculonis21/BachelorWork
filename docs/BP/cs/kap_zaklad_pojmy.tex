%%% Fiktivní kapitola s ukázkami sazby

\chapter{Základní pojmy}

V této kapitole vysvětlíme a rozebereme důležité pojmy, se kterými se v dalším
popisu práce budeme setkávat. Znalost těchto pojmů je potřebná pro pochopení
důvodů volby daných vybraných technologií a pro pochopení základního rozboru
implementace řešení, kterou popíšeme v následujících kapitolách.

V této kapitole nejdříve vysvětlíme základní teorii evolučních algoritmů
\ref{Evoluční algoritmy} a ukážeme si již existující knihovny pracující nebo
pomáhající pracovat s genetickými algoritmy \ref{Evoluční algoritmy -
implementace}. Poté se podíváme na základ teorie neuronových sítí \ref{NN} a
popíšeme pokročilejší evoluční algoritmy (NEAT \ref{NN - NEAT}, HyperNEAT
\ref{NN - HyperNEAT}) sloužící přímo k vývoji těchto sítí. Dále popíšeme
simulátory prostředí \ref{Simulované prostředí} a fyzikální simulátory
\ref{Simulované prostředí - f simulátory}, které využijeme pro simulaci při
vyvíjení našich robotů.

\section{Evoluční algoritmy} \label{Evoluční algoritmy}

TODO: POPIS EVOLUČNÍ ALGORITMY

\subsection{Existující implementace} \label{Evoluční algoritmy - implementace}

Pro vývoj řízení robotů budeme využívat evoluční algoritmy. Naše knihovna tedy
bude implementovat několik alespoň základních genetických operátorů,
používaných při vývoji jedinců a co nejjednodušeji umožnit jejich konfiguraci před
spouštěním jednotlivých experimentů. Naším cílem je co možná nejvíce
zpřístupnit knihovnu, která má být výsledkem této práce, aby uživatel se základní
znalostí genetických algoritmů a programovacího jazyka byl schopný pochopit běh
algoritmu a v případě potřeby mohl jednoduše provádět zásahy do jeho běhu. Není
pro nás tedy nutné najít tu nejefektivnější knihovnu, nýbrž tu, která přinese
výhody jako přehlednost a snadnou úpravu algoritmů, bez větších obtíží s
implementací a pochopením knihovny.

Dále představíme několik knihoven implementujících nebo usnadňujících
implementaci genetických algoritmů nebo jejich částí. 

\begin{itemize}
    \item \textbf{DEAP}\\ DEAP (\emph{Distributed Evolutionary Algorithms in
        Python}) je open-source Python knihovna pro rychlou tvorbu a
        prototypování evolučních algoritmů, která se snaží jejich tvorbu
        zjednodušit pomocí přímočarého postupu, podobného pseudokódu, který je
        se základní znalostí knihovny poměrně jednoduchý na porozumění Knihovna
        je tvořena ze dvou hlavních struktur \texttt{creator}, který slouží k
        vytváření genetických informací jedinců z libovolných datových struktur
        a \texttt{toolbox}, který je seznamem nástrojů (genetických operátorů),
        které mohou být použité při sestavování evolučního algoritmu. Dalšími
        menšími strukturami jsou \texttt{algorithms} obsahující 4 základní typy
        algoritmů a \texttt{tools} implementující další základní operátory
        (kusy operátorů), které je posléze možné přidávat do \texttt{toolbox}.
        Pomocí těchto základních stavebních bloků mohou uživatelé poměrně
        jednoduše začít tvořit skoro libovolné evoluční algoritmy
        \citep{fortin2012deap}. Následuje ukázka kódu tvorby základních částí
        evolučního algoritmu pro One Max problém, popsaná v oficiální
        dokumentaci knihovny DEAP \citep{deapproject}. V problému One Max máme
        populaci jedinců tvořených z jedniček a nul a chceme vyvinout jedince,
        který má na všech pozicích jen samé jedničky.

\begin{code}
import random

from deap import base
from deap import creator
from deap import tools
\end{code}

        Tvorba vlastních tříd fitness funkce a individuí pomocí
        \texttt{creator}.
\begin{code}
creator.create("FitnessMax", base.Fitness, weights=(1.0,))
creator.create("Individual", list, fitness=creator.FitnessMax)
\end{code}

        Šablonu vytvořených individuí můžeme dále použít při tvorbě populace
        následujícím stylem.
\begin{code}
toolbox = base.Toolbox()

toolbox.register("attr_bool", random.randint, 0, 1)
toolbox.register("individual", tools.initRepeat, 
                 creator.Individual, toolbox.attr_bool, 100)
toolbox.register("population", tools.initRepeat, 
                 list, toolbox.individual)
\end{code}
        
        Zde jsme vytvořili dvě inicializační funkce \texttt{individual()} a
        \texttt{population()}, které když zavoláme, vytvoří novou instanci
        individua nebo populace.

        V tomto problému je evaluační funkce jednoduchá, vytvoříme ji
        následovně.
\begin{code}
def evalOneMax(individual):
    return sum(individual)
\end{code}
        
        Pro zvolení genetických operátorů je musíme registrovat v
        \texttt{toolbox}.
\begin{code}
toolbox.register("evaluate", evalOneMax)
toolbox.register("mate", tools.cxTwoPoint)
toolbox.register("mutate", tools.mutFlipBit, indpb=0.05)
toolbox.register("select", tools.selTournament, tournsize=3)
\end{code}
        
        Nyní je vše připravené a můžeme začít s evolucí populace. Nejdřív
        populaci vygenerujeme a poté populaci vyhodnotíme.
\begin{code}
pop = toolbox.population(n=300)
fitnesses = list(map(toolbox.evaluate, pop))
for ind, fit in zip(pop, fitnesses):
    ind.fitness.values = fit

# Extracting all the fitnesses of 
fits = [ind.fitness.values[0] for ind in pop]
\end{code}

        Následně začneme evoluci. Naši jedinci jsou tvořeni ze 100 čísel 0/1.
        Evoluce poběží tak dlouho, dokud nebude existovat jedinec z populace,
        který má všech 100 čísel nastavených na 1, nebo evoluce doběhne
        určitého počtu generací.
\begin{code}
# Number of generations
g = 0
# Begin the evolution
while max(fits) < 100 and g < 1000:
    g = g + 1
\end{code}
        
        Nejdříve na základě fitness vybereme určité jedince jako rodiče další
        generace.
\begin{code}
    # Select the next generation individuals
    offspring = toolbox.select(pop, len(pop))
    # Clone the selected individuals
    offspring = list(map(toolbox.clone, offspring))
\end{code}

        Následně na vybrané jedince aplikujeme operátory křížení a mutace.
\begin{code}
    # Apply crossover and mutation on the offspring
    for child1, child2 in zip(offspring[::2], offspring[1::2]):
        if random.random() < CXPB:
            toolbox.mate(child1, child2)
            del child1.fitness.values
            del child2.fitness.values

    for mutant in offspring:
        if random.random() < MUTPB:
            toolbox.mutate(mutant)
            del mutant.fitness.values
\end{code}
        \texttt{CXPB} a \texttt{MUTPB} jsou pravděpodobnosti aplikování operátorů
        křížení a mutace. Klíčové slovo \texttt{del} udělá hodnotu fitness
        daného jedince neplatnou.

        Dále necháme zopakovat evaluaci celé populace jedinců a proces se může
        opakovat.
        
        Dle zdrojů popisující porovnání několika Python modulů, snažící se
        ulehčit práci s evolučními algoritmy, je DEAP nejefektivnější, tedy že 
        tvoří nejkratší kód, v porovnání počtu řádků potřebných pro tvorbu
        algoritmu řešící One Max problém z ukázky \citep{fortin2012deap}.

    \item \textbf{Inspyred}\\
        Inspyred poskytuje většinu z nejpoužívanějších evolučních algoritmů a
        dalších přírodou inspirovaných algoritmů (simulace reálných
        biologických systémů - př. optimalizace mravenčí kolonií) v jazyce
        Python. Knihovna je již předpřipravená s funkčním řešením, ve formě
        jednotlivých komponentů (Python funkcí), které si uživatel může sám
        upravovat, nebo je úplně nahradit za vlastnoručně vytvořené řešení v
        podobě vlastních funkcí. Uživatel pak při tvorbě algoritmu definuje
        několik struktur, které ovlivňují jak celý vývoj probíhá. Těmito
        strukturami jsou - struktury specifické k danému řešenému problému
        \texttt{generator} (jak jsou generována řešení = jedinci) a
        \texttt{evaluator} (definice fitness funkce)). A struktury specifické k
        danému evolučnímu algoritmu - \texttt{observer} (jak uživatel
        monitoruje evoluci), \texttt{terminator} (definuje pravidla pro konec
        evoluce), \texttt{selector} (kteří jedinci se mají stát rodiči),
        \texttt{variator} (jak jsou potomci vytvořeni z aktuálních jedinců),
        \texttt{replacer} (volí kteří jedinci mají přežít do další generace),
        \texttt{migrator} (jak se přenáší jedinci mezi různými
        populacemi/generacemi) a \texttt{archiver} (jak jsou jedinci ukládání
        mimo stávající populaci). Libovolný vybraný z těchto komponentů pak může být
        nahrazen odpovídající vlastní implementací \citep{tonda2020inspyred}.

\end{itemize}

\section{Neuronové sítě} \label{NN}
\section{NEAT} \label{NN - NEAT}
\section{HyperNEAT} \label{NN - HyperNEAT}

\section{Simulované prostředí} \label{Simulované prostředí}

Jelikož chceme vyvíjet řízení robotů založených na korektních fyzikálních
pravidlech a interakcích, je pro tuto práci důležité vybrat vhodný simulátor
prostředí. Přáli bychom si mít možnost jednoduše konfigurovat co nejvíce
vlastností prostředí a zároveň mít co nejlehčí přístup k morfologii
simulovaných robotů. Zároveň chceme, abychom měli možnost do morfologie robotů
nějakým stylem zasahovat i v průběhu vývoje a aktivně ji za běhu měnit. Jelikož
plánujeme v prostředí provádět experimenty s různými typy robotů, používající
různé styly pohybu (typy motorů, kloubů, tvarů končetin, atd.), je potřebné,
aby fyzikální simulátor (\emph{fyzikální řešič=solver}) byl schopný simulovat i
složitější typy robotů. Takovými mohou být právě třeba kráčející roboti, neboli
roboti používající k pohybu končetiny připomínající nohy, na rozdíl od
jednodušších typů robotů, kteří se mohou pohybovat pomocí kol, jejichž simulace
bývá mnohdy jednodušší. Stejně tak jak potřebujeme umožnit složitost robotů,
protože nebudeme mít možnost vlastnoručně kontrolovat každý parametr, který
bude při vývoji robotům přiřazen, potřebujeme zajistit, aby fyzikální simulátor
zvládal libovolné rozsahy parametrů a simulace zůstal pro tyto parametry
stabilní. Zároveň chceme, aby simulátor v prostředí byl
deterministický, což umožní, že předváděné experimenty můžeme dle potřeby
opakovat a výsledky tak náležitě prezentovat. Evoluční algoritmy jsou velmi
lehce paralelizovatelné a tedy pro urychlení procesu vývoje a experimentů nám
bude výhodné, pokud by simulace zvládala paralelní běh na více vláknech (více
simulací, každá na vlastním vlákně). V posledním řadě pro lehčí integraci do
vlastního modulu bude užitečné, aby modul spravující zvolený simulátor byl
open-source, což nám dá volnost v případě, že si budeme chtít chování systémů v
prostředí nějak vlastnoručně upravit.

Při hledání simulátorů prostředí, které by vyhovovali našim požadavkům a
umožňovali kontrolu a ovládání prostředí skrz zvolený jazyk Python, jsme
narazili na několik možností.

\begin{itemize}
    \item \textbf{Gazebo}\\
        Gazebo je sada open-source víceplatformní knihoven pro vývoj, výzkum a
        aplikaci robotů, původně založená v roce 2002. Umožňuje kompletní
        kontrolu nad simulací dynamického 3D prostředí s více agenty a
        generování dat ze simulovaných senzorů. Fyzikálně korektní interakce v
        prostředí pak od začátku projektu zajišťuje známý fyzikální simulátor
        ODE~\ref{ODE}, nad kterým Gazebo tvoří abstraktní vrstvu, umožňující
        snazší tvorbu simulovaných objektů různých druhů. V dnešní době je
        stále výchozím fyzikálním simulátorem ODE, nicméně uživatel již může
        vybrat celkem ze čtyř různých simulátorů - Bullet~\ref{Bullet},
        Simbody, Dart~\ref{Dart} a ODE. Uživatel s knihovnou pracuje skrz
        grafické rozhraní založené na knihovně Open Scene Graph používající
        OpenGL, nebo skrz příkazovou řádku. Prostředí a roboti mohou být
        tvořené buď skrz grafické prostředí, nebo v textovém formátu XML.
        Limitací Gazebo je pak neschopnost rozdělit simulace mezi vícero vláken
        kvůli vnitřní architektuře spojené s fyzikální simulací
        \citep{koenig2004design}. 

    \item \textbf{Webots}\\
        Webots je open-source víceplatformní robustní a deterministický
        robotický simulátor vyvíjený od roku 1998, umožňující programování a
        testování virtuálních robotů mnoha různých typů a jednoduchou následnou
        aplikaci softwaru na reálné roboty. Simulátor je možné použít pro
        simulaci prostředí s vícero agenty najednou s možnostmi lokální i
        globální komunikace mezi agenty. Výpočty fyzikálních interakcí
        zajišťuje fyzikální simulátor ODE. Pro vývoj robotů a
        prostředí je možné využít řady programovacích jazyků a to C, C++,
        Python, Java, MATLAB nebo ROS (\emph{Robot Operating System}).
        Prostředí umožňuje práci v grafickém rozhraní a vizualizaci simulací
        pomocí OpenGL. Knihovna dále nabízí využití připravených modelů robotů,
        vlastní editor robotů a map a možnosti vložení vlastních robotů z 3D
        modelovacích softwarů v CAD formátu \citep{michel2004cyberbotics}
        \citep{Webots}.

    \item \textbf{CoppeliaSim}\\ CoppeliaSim (kdysi známý pod jménem
        \emph{V-REP} = \emph{Virtual Robot Experimentation Platform}) je
        víceplatformní simulační modul pro vývoj, testování a jednoduchou
        aplikaci softwaru pro roboty. Dovoluje vývoj ovladačů pomocí 7 různých
        programovacích jazyků a ulehčuje jejich aplikace v simulovaných a
        skutečných robotech. Simulaci ovladačů je možno jednoduše
        rozdistribuovat mezi vícero vláken dokonce vícero strojů, což urychluje
        vývoj a snižuje nároky na procesor v době simulace. Navíc je možné
        vyvíjený ovladač nechat v době simulací běžet na vlastním na dálku
        připojeném robotovi, co dále ulehčuje přenos finální verze ovladačů od
        vývoje do skutečného světa. Prostředí umožňuje práci s širokou řadou
        typů objektů, druhů kloubů, senzorů a dalších objektů obvykle
        používaných při vývojích robotických ovladačů. Obsahuje lehce
        použitelný editor prostředí a robotů samotných s řadou předem
        vytvořených modelů, které může uživatel hned využít. Modely zároveň
        mohou být přidány skrz řadu různých formátů (XML, URDF, SDF). Prostředí
        podporuje pět různých fyzikálních simulátorů (Bullet, ODE,
        MuJoCo~\ref{MuJoCo}, Vortex~\ref{Vortex}, Newton), mezi kterými si
        uživatel může vybrat dle potřeb přesnosti (reálnosti), rychlosti a
        dalších možností jednotlivých fyzikálních simulátorů
        \citep{coppeliaSim} \citep{nogueira2014comparative}.
\end{itemize}

\subsection{Fyzikální simulátory} \label{Simulované prostředí - f simulátory}

V této podkapitole se podíváme na základní popis a možné výhody a nevýhody
jednotlivých fyzikálních simulátorů, na které jsme narazili při hledání
simulátorů prostředí.

\begin{itemize}
    \item \textbf{ODE}\\ \label{ODE}
        ODE (\emph{Open Dynamics Engine}) je víceplatformní open-source
        fyzikální simulátor, jehož vývoj začal v roce 2001. Vhodný pro simulaci
        pevných těles s různými druhy kloubů a pro detekci kolizí. Tvořený pro
        využití v interaktivních nebo real-time simulacích, upřednostňující
        rychlost a stabilitu nad fyzikální přesností \citep{smith2007open}.
        Potřeba menších simulačních kroků pro stabilitu. Hodí se pro simulaci
        vozidel, pochodujících robotů a virtuálních prostředí. Široké využití v
        počítačových hrách a 3D simulačních nástrojích
        \citet{coppeliarobotics}.

    \item \textbf{Bullet}\\ \label{Bullet}
        Bullet je open-source fyzikální knihovna, podporující detekci kolizí a
        simulaci pevných a měkkých těles. Bullet je používán jako fyzikální
        simulátor pro hry, vizuální efekty a robotiku \citep{coumans}. Byl použit jako hlavní
        fyzikální simulátor pro simulaci NASA \emph{Tensegrity} robotů (s
        vlastními úpravami pro simulaci měkkých těles, kvůli nerealistickým
        metodám řešení simulace provazů) \citep{izadi2018simulating}.

    \item \textbf{Dart}\\ \label{Dart}
        Dart (\emph{Dynamic Animation and Robotics Toolkit}) je víceplatformní
        open-source knihovna pro simulace a animace robotů. Od předchozích se odlišuje
        stabilitou a přesností, díky zobecněné reprezentaci koordinací pevných
        těles v simulaci. Na rozdíl od ostatních fyzikálních simulátorů, aby
        dal vývojáři plnou kontrolu nad simulací, umožňuje Dart plný přístup k
        interním hodnotám simulace. Zároveň se díky línému vyhodnocování hodí
        pro vývoj real-time ovladačů pro roboty. \citep{lee2018dart}.

    \item \textbf{MuJoCo}\\ \label{MuJoCo}
        MuJoCo (\emph{Multi-Joint Dynamics with Contact}) je open-source
        fyzikální simulátor pro vývoj v oblasti robotiky, biomechaniky a
        dalších. Často je využíváno pro testování a porovnávání různých metod
        navrhování robotických systémů jako jsou třeba evoluční algoritmy nebo
        metody zpětnovazebného učení \citep{salimans2017evolution}. V
        simulacích je pro roboty možné nakonfigurovat využití mnoha druhů
        aktuátorů, včetně těch simulující práci svalů a k dispozici je i velké
        množství kloubů. Simulátor zároveň umožňuje velký nárůst v rychlosti
        běhu simulace za pomoci plné podpory paralelizace na všech dostupných
        vláknech a stabilitě simulace i při velmi velkých simulačních krocích
        \citep{todorov2014mujoco}. Zároveň nabízí jednoduchý styl, jakým si
        může uživatel konfigurovat všechny detaily simulace a samotných
        simulovaných robotů pomocí jednoduchých XML konfiguračních souborů (XML
        formát modelů \emph{MJCF}). V komplexním rozboru řady četně používaných
        fyzikálních simulátorů byl simulátor MuJoCo hodnocen jako jeden z
        nejlepších co se týče stability, přesnosti a rychlosti simulací. Další
        výhodou zlepšující přesnost tohoto simulátoru je, že MuJoCo pro
        simulaci používá kloubní souřadnicový systém, který předchází narušení
        fyzikálních pravidel a tedy nepřesností v kloubech
        \citep{erez2015simulation}.

    \item \textbf{Vortex}\\ \label{Vortex}
        Vortex je uzavřený, komerční fyzikální simulátor určený pro tvorbu
        reálnému světu odpovídajících simulací. Obsahuje mnoho parametrů,
        umožňující nastavení reálných fyzikálních parametrů dle potřeb,
        většinou industriálních a výzkumných aplikací \citep{coppeliarobotics}
        \citep{yoon2023comparative}.

\end{itemize}
