\chapter*{Úvod}
\addcontentsline{toc}{chapter}{Úvod}

V dnešní době stále přibývá možností, kde se snažíme aplikovat metody umělé
inteligence pro řešení různorodých problémů. Na řadu z těchto problémů se nám
může nabízet hned několik možných řešení. Problém ale může nastat, pokud si
nejsme jistí, nebo třeba vůbec není možné přesně definovat, co vlastně by mělo
být správným řešením.

\paragraph{}
Pro tyto problémy se nám často hodí využívat metod evolučních algoritmů. Jedná
se o přírodou inspirované optimalizační algoritmy -- konkrétně Darwinovou
evoluční teorií -- které napodobováním přírodních procesů hledají dle našich
požadavků ta nejlepší řešení.

Zacházení s těmito algoritmy ale nemusí být vůbec jednoduché a podobně jako u
dalších optimalizačních metod a metod strojového učení je jejich běh zahalen
množstvím parametrů, které spolu souvisí často špatně předvídatelným způsobem.

\paragraph{}
Z tohoto důvodu je cílem této práce vytvořit platformu, která bude přístupná
uživatelům různých úrovní specializace, umožňující tvořit a provádět
experimenty s evolučními algoritmy. 

S tímto cílem volíme tvořit experimenty s roboty ve virtuálním prostředí,
což~se na tento problém velmi dobře hodí. Uživatel díky robotům intuitivně
chápe složitost problému a každý posun v řešeném problému je interaktivně
pozorovatelný v daném prostředí ještě v průběhu hledání řešení.

Cílem projektu je, aby uživatel dostal kontrolu nad experimenty a mohl tak
získat lepší přehled o práci s evolučními algoritmy a pochopil tak množství
parametrů a jejich vzájemné souvislosti, se kterými se můžeme při tvorbě
experimentů setkat. Projekt bude obsahovat různorodou řadu problémů, na které
bude potřeba využít vícero různých přístupů, což umožní dále rozšířit pochopení
problémů evolučních algoritmů. Nejtěžšími pak mohou být problémy, vyžadující
využití neuronových sítí, což může být pro uživatele díky tomuto projektu
jednoduchým, prvním využitím pokročilého algoritmu pro neuroevoluci (evoluční
vývoj neuronových sítí) -- NEAT.

Pro lehce pokročilého uživatele bude projekt dále nabízet možnosti nahlédnutí
do útrob projektu, prezentující, jak se s evolučními algoritmy zachází v
samotném zdrojovém kódu, což mu zároveň umožní tvořit vlastní pokročilé
experimenty a~upravovat a rozšiřovat připravenou databázi částí evolučních
algoritmů o vlastní.

\paragraph{}
Práce je rozdělena do čtyř hlavních kapitol. V kapitole
\ref{chapter-zakladní pojmy} si představíme a~vysvětlíme základní pojmy
využívané v této práci jako jsou evoluční algoritmy, neuronové sítě a další.
Dále v kapitole \ref{chapter-specifikace} blíže popíšeme specifikaci projektu
a~upřesníme jakých cílů tímto projektem chceme dosáhnout. V kapitole
\ref{chapter-implementace} již rámcově představíme jak je celý projekt interně
poskládaný a poskytneme tak základní náhled na~to, jak knihovna pracuje uvnitř.
V poslední kapitole \ref{chapter-experimenty} na příkladech předvedeme
typy experimentů, které si uživatel bude moci sám ve výchozí verzi vyzkoušet, a
které bude mít možnost hned upravovat a testovat. Zároveň v této kapitole
ukážeme způsob, jakým tyto experimenty jdou statisticky vyhodnocovat
a~rozebereme výsledky ukázkových experimentů. V závěru pak bude následovat
shrnutí výsledků práce a budou navržena možná rozšíření projektu. V příloze
čtenář nalezne uživatelskou dokumentaci projektu. Programátorská dokumentace a
zdrojové kódy jsou součástí elektronické přílohy práce a lze je také nalézt v
GitLabu
(\url{https://gitlab.mff.cuni.cz/teaching/nprg045/mraz/becvar2022.git}).
