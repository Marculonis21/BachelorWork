\chapter*{Závěr}
Práce představuje platformu pro tvorbu a provádění experimentů s evolučními
algoritmy na simulovaných robotech. Práce splnila cíle, které pro ni byly
stanoveny. Vytvořili jsme systém přístupný uživatelům s různými úrovněmi
pochopení problematiky evolučních algoritmů a zpřístupnili tak nástroj, pomocí
kterého budou uživatelé moci jednoduše prohlubovat a aplikovat své znalosti.
Platforma nabízí řadu implementovaných operátorů evolučních algoritmů a několik
robotů různých úrovní složitosti, se kterými může uživatel okamžitě pracovat.

Při tvorbě platformy jsme dbali na to, aby zdrojový kód byl srozumitelný
a~jednotlivé části logicky rozdělené, což uživatelům umožní a zjednoduší
přístup ke zdrojovému kódu. Tím nabízíme možnost hlubšího pochopení fungování
evolučních algoritmů a jejich případného rozšiřování a aplikování nových
vlastních částí.

Experimenty provedené v této práci ukazují jak typy experimentů,
které~platforma umožňuje, tak styly jejich vyhodnocení. Pomocí experimentů jsme
na dvou robotech různých složitostí ověřili fakt, že jednoduché evoluční
algoritmy nestačí pro řešení všech problémů, a že pro určité pokročilé problémy
je třeba pokročilých evolučních algoritmů. 

Zároveň jsme demonstrovali příklady experimentů, které umožňují vývoj jak
řízení, tak morfologie robotů. Pro tento problém jsme v práci navrhli a
implementovali způsob, jakým je možné za běhu algoritmů měnit XML konfigurační
soubory robotů z knihovny \emph{MuJoCo}. Tyto experimenty předvedly zajímavé
výsledky, vytvářející roboty různých konfigurací, které byly schopné dosáhnout
lepších výsledků než ty výchozí.

Práce nabízí mnoho možností pro rozšiřování. Vedle pokročilejších evolučních
algoritmů na vývoj řízení robotů by zajímavým rozšířením mohlo být prozkoumání dalších
možností vývoje morfologie, umožňující rozsáhlejší změny v konfiguraci robotů.
Dalším by mohlo být podrobnější zkoumání dopadů různých fitness funkcí na
výsledky evolučního vývoje.

\addcontentsline{toc}{chapter}{Závěr}
