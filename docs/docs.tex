\documentclass[a4paper, 12pt]{article}
\usepackage[T1]{fontenc}
\usepackage[utf8]{inputenc}
\usepackage{booktabs}
\usepackage{titling}
\usepackage{titlesec}
\usepackage{amssymb}
\usepackage{pifont}
\usepackage{graphicx}
\graphicspath{ {../../} }

\usepackage{hyperref}
\hypersetup{
    colorlinks=true,
    linkcolor=black,
    urlcolor=cyan,
}
\urlstyle{same}

\renewcommand{\contentsname}{Obsah}
\renewcommand{\thesection}{\Roman{section}}
\renewcommand{\thesubsection}{\roman{subsection}}

\titleformat{\section}
{\Large\bfseries}
{\thesection}
{0.5em}
{}


\titleformat{\subsection}
{\large\bfseries}
{\thesubsection.}
{0.5em}
{}

\title{
        \vspace{1in}
        \rule{\linewidth}{0.5pt}
		\usefont{OT1}{bch}{b}{n}
        \huge Dokumentace ročníkového projektu \\\vspace{20pt}Vývoj řízení
        simulovaných robotů ve 3D prostředí
        \vspace{-10pt}
        \rule{\linewidth}{1pt}
}
\author{
		\normalfont\normalsize
        Marek Bečvář\\\normalsize
        MFF UK 2022
}
\date{}

\begin{document}
\maketitle 
\newpage

\tableofcontents
\newpage

\section{Popis a cíl projektu} 
\paragraph{}
Projekt je zaměřen na využití genetických algoritmů pro vývoj řízení simulovaných 
robotů ve fyzikálním prostředí. Řízení se má vyvíjet směrem k předem
specifikovanému cíli. Tímto cílem může být mimo jiné například uražená vzdálenost nebo
maximální rychlost pohybu.
Cílem projektu je seznámení se s různými možnostmi zvoleného fyzikálního
prostředí, vývoj základního genetického algoritmu a jeho aplikace na sadu
výchozích a vlastních simulovaných robotů a vytvoření další sady aplikací
umožňující statistické zpracování výsledků. 

\section{Dostupné technologie}
\subsection{Fyzikální simulátory}
\paragraph {MuJoCo} (\emph{Multi-Joint Dynamics with Contact}) je zdarma a open 
source robustní fyzikální engine pro vývoj v oblasti robotiky, biomechaniky a dalších.
MuJoCo se dále často využívá pro testování a porovnávání různých metod
navrhování robotických systémů jako jsou třeba evoluční algoritmy nebo reinforcement
learning \cite{salimans2017evolution}.

MuJoCo umožňuje velký nárůst v rychlosti běhu simulace za pomoci plné podpory
paralelizace na všech dostupných jádrech a stabilitě simulace i při využití
větších simulačních časových kroků \cite{todorov2012mujoco}. Zároveň nabízí jednoduchý styl,
jakým si může uživatel upravit všechny detaily simulace i robotů samotných
pomocí C++ API nebo jednoduchých XML konfiguračních souborů. 

\paragraph{Webots}
Webots je open source multiplatformní robotický simulátor s \\využitím v
průmyslu, výuce i výzkumu \cite{Webots}. Webots umožňuje programování a
testování jak jednoduchých virtuálních robotů, tak následné aplikace softwaru na 
reálné roboty, obojí využívající programovacího jazyka C a stejných \emph{Khepera}
API\cite{michel1998webots}. 

Prostředí nabízí využití řady předpřipravených modelů robotů všech druhů 
a dále umožňuje import vlastní robotů z 3D modelovacích softwarů v CAD formátu.

\paragraph{Player/Stage}
Player/Stage je výsledkem projektu \emph{The Player Project}
\cite{playerproject} vytvořeným za účelem usnadnit vývoj v oblasti robotiky 
a senzorů. Player/Stage je spojení dvou projektů. 

\textbf{Player} je síťový jazykově a na platformě nezávislý 
server pro ovládání robota s jednoducým přístupem k senzorům a motorům robota
skrz IP síť. Klientský program může být vytvořený a spuštěný na libovolném PC s
připojením k síti robota a v libovolném jazyce podporujícím TCP sokety. 

\textbf{Stage} je rychlý a rozšířitelný 2D simulátor prostředí s objekty a roboty 
s jejich senzory, kam roboti a jejich ovladače mohou být načítány za běhu
simulace.

\textbf{Player/Stage} Často jsou tyto dva programy využívané dohromady tak, že
uživatel vyvine populaci robotů (ovladačů, senzorů) a poskytne ji jako klienty
pro Player server.

\paragraph{Gazebo}
Gazebo je sada open source knihoven pro vývoj, výzkum a aplikaci robotů. 
Umožňuje simulaci dynamického 3D prostředí s více agenty, generování reálných
dat ze simulovaných senzorů a fyzikálně korektní interakce robotů s
prostředím \cite{gazebo1389727}. Pro simulace umožňuje výběr z více fyzikálních
enginů. 

Uživatel pracuje a nastavuje prostředí v interním grafickém prostředí s určitou možností 
spouštět simulace i bez GUI.

Gazebo bylo součástí projektu \emph{Player Project} od roku 2004, ale od roku 2012
je nezávislým projektem \emph{Open Robotics} \cite{gazebo}
\cite{playerproject}.

\subsection{Evoluční algoritmy}
\paragraph{DEAP}
DEAP (\emph{Distributed Evolutionary Algorithms in Python}) je Python framework
pro tvorbu evolučních algoritmů, který se snaží jejich tvorbu zjednodušit pomocí
přímočarého postupu (podobného pseudokódu), který je jednoduchý na porozumění.
\cite{fortin2012deap}

Framework je tvořený ze dvou hlavních struktur \emph{creator}, který pomáhá s
vytvářením genetických informací z jedinců z libovolných datových struktur a
\emph{toolbox}, který je seznamem nástrojů (genetických operátorů), které mohou
být použité při tvorbě algoritmu. 

\paragraph{Inspyred}
Inspyred poskytuje implementaci většiny z nejpoužívanějších evolučních
algoritmů a dalších přírodou inspirovaných algoritmů v jazyce Python. \cite{tonda2020inspyred}

Knihovna již přichází s funkčním řešením, ve formě jednotlivých komponentů
(python funkcí), které ale uživatel může sám přepsat, nebo nahradit za jiné již
vytvořené funkce.

\newpage

\section{Použité technologie}
\begin{itemize}
    \item MuJoCo 
    \item Vlastní evoluční algoritmus
    \item OpenAI - Gym (Python API pro vývoj AI v různých prostředích)
    \item Python (Programovací jazyk)
    \item XML
\end{itemize}


\paragraph{OpenAI - Gym} OpenAI je firma zaměřená na výzkum, vývoj a praktické využití umělé inteligence.
\textbf{Gym} je open source Python API firmy OpenAI. Je to platforma pro vývoj převážně
reinforcement learning metod \cite{brockman2016openai}. Umožňuje využít řadu prostředí, 
ve kterých uživatelé mohou jednoduše spouštět a testovat své agenty \cite{brockman2016openai}.
Tato prostředí mohou být \\ku příkladu Atari hry, textové hry, jednoduché 2D i plně fyzikálně simulované 3D
prostředí (s fyzikálně enginem \textbf {MuJoCo}).

Gym nabízí jednoduchý přístup do všech těchto prostředí kde vstupy (akce agenta
v prostředí) i výstupu (stav prostředí, pozorování agenta) jsou standardizované
napříč všemi prostředími. Navíc open source vlastnost \\tohoto API umožňuje
vlastní doprogramování pokročilých pomocných nástrojů pro vývoj a práci s 
prostředím.

I když je Gym primárně vytvořené pro vývoj reinforcement learning agentů, je
velmi jednoduché použít namísto toho agenta, který je v našem případě
vyvíjen pomocí evolučních algoritmů.

\subsection{Odůvodnění vybraných technologií}
\paragraph{Fyzikální simulátor - MuJoCo} MuJoCo bylo oproti ostatním
simulátorům zvoleno, kvůli jednoduchosti konfigurace celého prostředí. \\Dále
spojení s prostředím skrz open source API od OpenAI umožňuje přímočaré
upravování všech částí prostředí, což velmi pomáhá při vývoji vlastních
evolučních agentů. Dále, jelikož evoluční algoritmy jsou zdlouhavým procesem,
rychlost a možná paralelizace běhu prostředí byla při výběru \\pozitivní
vlastností. S ohledem na možné budoucí experimenty zároveň bylo žádoucí mít
možnost jednoduché úpravy všech aspektů simulovaných robotů a jejich prostředí.
Zde XML konfigurační soubory, které MuJoCo používá, pomocí kterých může uživatel 
postavit celého robota a zároveň skrz ně \\nastavit kompletní vlastnosti prostředí 
byly přínosem pro rychlejší experimentování a vývoj.

\paragraph{Využití vlastního evolučního algoritmu} Pro jednoduchost
implementace a snížení nároků na znalosti mnohdy velmi složitých externích
knihoven pro evoluční algoritmy, které krom zkrácení zápisu často nepřináší
více dalších pozitivních vlastností, bylo rozhodnuto o využití vlastní
jednoduché implementace všech základních operátorů evolučních algoritmů, které
je \\možné poté v implementaci vlastních experimentálních agentů jednoduše
\\aplikovat a používat bez potřeby znalostí externích knihoven.

\paragraph{Programovací jazyk - Python} Pro implementaci všeho je využitý 
programovací jazyk Python. Důvodem byla potřeba nějakého jednoduchého, nejlépe
interpretovaného jazyka, vhodného pro rychlé experimentování a dobré pro
předvádění finálních výsledků.

\paragraph{Odkazy} 
\begin{itemize}
    \item \href{https://www.gymlibrary.ml/}{Oficiální Gym library docs}
    \item \href{https://github.com/openai/gym}{Gym Github} 
    \item \href{https://blog.paperspace.com/getting-started-with-openai-gym/}{Getting started with OpenAI Gym} 
    \item \href{https://medium.com/velotio-perspectives/exploring-openai-gym-a-platform-for-reinforcement-learning-algorithms-380beef446dc}{Medium článek - Exploring OpenAI Gym}
    \item \href{https://www.python.org/}{Python.org}
    \item \href{https://en.wikipedia.org/wiki/Python_(programming_language)}{Python - Wikipedia} 
    \item \href{https://developer.mozilla.org/en-US/docs/Web/XML/XML_introduction}{XML Introduction - Mozilla Developer} 
    \item \href{https://en.wikipedia.org/wiki/XML}{XML - Wikipedia}
\end{itemize}

\section{Popis softwarového díla}

\paragraph{Průběh experimetnu} Nejprve je potřeba pomocí XML souboru zvolit určitého 
definovaného robota pro náš experiment. Dále mu přiřadíme \\vytvořeného agenta,
který ovlivňuje jak jsou do genetického algoritmu kódovány vstupy z prostředí
libovolného robota. Tento agent má předem přiřazené genetické operátory, které
ale uživatel může dle potřeb experimentu zaměňovat za jiné, buď předem
definované, nebo zcela vlastní vytvořené \\operátory ve tvaru odpovídajících
Python funkcí.
Dále máme možnost takto připravenému prostředí nakonfigurovat vlastní fitness
funkci a nastavit \\chtěnou dobu trvání experimentu (počet generací, které má
algoritmus projít). Rozběhlé prostředí jsme pak schopni pozorovat pomocí
aktivních statistik, které nám dávají představu o hodnotách z vývoje v 
prostředí. Pokud uživatel chce mít větší přehled o vývoji v prostředí, má 
možnost určité generace (jedince z generace) interaktivně sledovat skrz kamery 
v simulovaném prostředí, ve kterém experiment probýhá. 
Jelikož experimenty pro dosažení nějakých přínosných výsledků je potřeba pro
kontrolu opakovat, umožňuje program opakované spouštění experimentů s předem
zvolenými parametry (úprava parametrů možna jak pro prostředí, tak pro
samotného agenta a jeho genetické operátory). Všechny tyto výsledky jsou pak
podle zvolených\\ parametrů ukládány, umožňující následné externí
zpracování, popřípadě využití jednoduchých statistických pomůcek (vytvořených v
rámci tohoto projektu) pro rozbor dat z výsledků běhů algoritmů v grafech.
Z každého běhu je zároveň ukládán nejlepší jedinec, pro umožnění vizuální
kontroly a rozboru výsledného jedince finální generace.

\bibliographystyle{unsrt}
\bibliography{references}

\end{document}
