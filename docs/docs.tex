\documentclass[a4paper, 12pt]{article}
\usepackage[T1]{fontenc}
\usepackage[utf8]{inputenc}
\usepackage{booktabs}
\usepackage{titling}
\usepackage{titlesec}
\usepackage{amssymb}
\usepackage{pifont}
\usepackage{graphicx}
\graphicspath{ {../../} }

\usepackage{hyperref}
\hypersetup{
    colorlinks=true,
    linkcolor=black,
    urlcolor=cyan,
}
\urlstyle{same}

\renewcommand{\contentsname}{Obsah}
\renewcommand{\thesection}{\Roman{section}}
\renewcommand{\thesubsection}{\roman{subsection}}

\titleformat{\section}
{\Large\bfseries}
{\thesection}
{0.5em}
{}


\titleformat{\subsection}
{\large\bfseries}
{\thesubsection.}
{0.5em}
{}

\title{
        \vspace{1in}
        \rule{\linewidth}{0.5pt}
		\usefont{OT1}{bch}{b}{n}
        \huge Dokumentace ročníkového projektu \\\vspace{20pt}Vývoj řízení
        simulovaných robotů ve 3D prostředí
        \vspace{-10pt}
        \rule{\linewidth}{1pt}
}
\author{
		\normalfont\normalsize
        Marek Bečvář\\\normalsize
        MFF UK 2022
}
\date{}

\begin{document}
\maketitle 
\newpage

\tableofcontents
\newpage

\section{Popis a cíl projektu} 
\paragraph{}
Projekt je zaměřen na využití genetických algoritmů pro vývoj řízení simulovaných 
robotů ve fyzikálním prostředí. Řízení se má vyvíjet směrem k předem
specifikovanému cíli. 

Cílem projektu je seznámení se s různými možnostmi zvoleného fyzikálního
prostředí, vývoj základního genetického algoritmu a jeho aplikace na sadu
výchozích a vlastních simulovaných robotů a vytvoření další sady \\aplikací
umožňující statistické zpracování výsledků. 

\section{Dostupné technologie}
\subsection{Fyzikální simulátory}
\paragraph {MuJoCo} (\emph{Multi-Joint Dynamics with Contact}) je free a open 
source robustní fyzikální engine pro vývoj v oblasti robotiky, biomechaniky a dalších.

MuJoCo umožňuje velký nárůst v rychlosti běhu simulace za pomoci plné podpory
paralelizace na všech dostupných jádrech a stabilitě simulace i při využití
větších simulačních časových kroků. Zároveň nabízí jednoduchý styl,
jakým si může uživatel upravit všechny detaily simulace i robotů samotných
pomocí C++ API nebo jednoduchých XML konfiguračních souborů. 

\paragraph{Webots}
Webots 
% \section{Použité technologie}
% \begin{itemize}
%     \item MuJoCo (Fyzikální engine, nadstavba pro Gym pro robotické simulace ve
%         3D prostředí)
%     \item OpenAI - Gym (Python API pro vývoj AI v různých prostředích)
%     \item Python (Programovací jazyk)
%     \item XML
% \end{itemize}


\paragraph{Odkazy} 
\begin{itemize}
    \item \href{https://mujoco.org/}{MuJoCo.org}
    \item \href{https://openai.com/blog/faster-robot-simulation-in-python}{OpenAI MuJoCo} 
    \item \href{https://neptune.ai/blog/installing-mujoco-to-work-with-openai-gym-environments}{MuJoCo - Instalace} 
    \item \href{https://mujoco.readthedocs.io/en/latest/overview.html}{MuJoCo - Docs} 
    \item \href{https://mujoco.readthedocs.io/en/latest/XMLreference.html}{MuJoCo - XML reference} 
\end{itemize}

\newpage

\subsection{OpenAI - Gym}
\paragraph{OpenAI} je firma zaměřená na vývoj a praktické využití umělé inteligence.  
\paragraph{Gym} je open source Python API firmy OpenAI. Je to platforma pro vývoj převážně
Reinforcement learning metod. Umožňuje využít řadu prostředí, ve kterých
uživatelé mohou jednoduše spouštět a testovat své agenty. Tato prostředí mohou
být různé Atari hry, textové hry, jednoduché 2D i plně fyzikálně simulované 3D
prostředí (\textbf {MuJoCo}).

Gym nabízí jednoduchý přístup do všech těchto prostředí kde vstupy (akce agenta
v prostředí) i výstupu (stav prostředí, pozorování agenta) jsou standardizované
napříč všemi prostředími. Navíc open source vlastnost \\tohoto API umožňuje
vlastní doprogramování pokročilých pomocných nástrojů pro vývoj a práci s 
prostředími.

I když je Gym primárně vytvořené pro vývoj Reinforcement learning agentů, je
velmi jednoduché použít namísto toho například agenta, který je v našem případě
vyvíjen pomocí genetických algoritmů.

\paragraph{Odkazy} 
\begin{itemize}
    \item \href{https://www.gymlibrary.ml/}{Oficiální Gym library docs}
    \item \href{https://github.com/openai/gym}{Gym Github} 
    \item \href{https://blog.paperspace.com/getting-started-with-openai-gym/}{Getting started with OpenAI Gym} 
    \item \href{https://medium.com/velotio-perspectives/exploring-openai-gym-a-platform-for-reinforcement-learning-algorithms-380beef446dc}{Medium článek - Exploring OpenAI Gym}
\end{itemize}

\subsection{Další odkazy}
\begin{itemize}
    \item \href{https://www.python.org/}{Python.org}
    \item \href{https://en.wikipedia.org/wiki/Python_(programming_language)}{Python - Wikipedia} 
    \item \href{https://developer.mozilla.org/en-US/docs/Web/XML/XML_introduction}{ XML Introduction - Mozilla Developer} 
    \item \href{https://en.wikipedia.org/wiki/XML}{XML - Wikipedia}
\end{itemize}

\section{Popis softwarového díla}
\subsection{Rozdělení}
\paragraph{}Projekt je rozdělený do více Python skriptů. Pro rychlejší iteraci při vývoji
je potřeba jemné rozdělení všech možných částí a různých typů operací
\\genetických algoritmů (s otestovanou správností) tak, aby bylo později jednoduché
poskládat agenty z těchto předdefinovaných částí. 

Dále je třeba mít skripty, které umožňují spouštění genetického algoritmu za
určitým cílem (fitness funkce), ve specifikovaném prostředí, s vybraným typem
robota a za použití specifického typu agenta a jeho parametrů a později mít možnost 
výsledky z běhu algoritmu ukládat pro pozdější zpracování. Zároveň pro
snadnější zpracování a vývoj je užitečné mít uložené i celého nejlepšího výsledného 
agenta pro vizuální rozbor a kontrolu jeho finálního výsledku.

Poslední částí je statistický skript, který zpracuje výsledky z jednoho nebo
více běhů genetického algoritmu a podle uložených informací z jednotlivých běhů
zanese jejich výsledky dle potřeby do tabulky, nebo grafu.

\end{document}
